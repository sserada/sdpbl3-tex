\documentclass[a4paper,9pt,twocolumn,dvipdfmx]{jarticle}
\usepackage{sdpbl}
\usepackage{amsmath}
\pagestyle{empty}

\begin{document}
\twocolumn[
2022年度 東京都市大学 SD PBL(3) 最終レポート

\vspace{3mm}

\begin{center}
\Large
タイトル(14ポイント、中央)
\end{center}

\begin{flushright}
学科、学籍番号、氏名(10ポイント、右寄せ)\\
クラス教員名/教室番号、チーム番号/チーム名:メンバー名1、メンバー名2、・・(10ポイント、右寄せ)
\end{flushright}

\vspace*{7mm}

]

\kanjiskip=.1zw plus 3pt minus 3pt
\xkanjiskip=.1zw plus 3pt minus 3pt

\section{はじめに}
\subsection{形式}
最終報告書(レポート)は、A4で2~5ページにまとめる。2段組とする。余白は上下左右 約20mm、行数は40~45行、段の幅は22字程度とする。

文字は、タイトル14ポイント(中央)、所属氏名など10ポイント(右寄せ)、本文9ポイントとする。
なお、読点は「、」または「,」、句点は「。」または「.」とし、本文の日本語のフォントは明朝体、英数字のフォントはCenturyとする。
図表のタイトルも統一すること。
必ず、本テンプレートに従って作成すること。

\subsection{提出}
所属学科の締め切り日、提出方法を守ること。

\section{適切な章立てをして論をすすめること}
\subsection{必要に応じて小見出しをつける}
□□□□□□□□□□□□□□□□□□□□□□□□□□□□□□□□□□□□□□□□□□□□□□□□□□□□□□
□□□□□□□□□□□□□□□□□□□□□□□□□□□□□□□□。
□□□□□□□□□□□□□□□□□□□□□□□□□□□□□□□□□□□□□□□□□□□□□□□□□□□□□□
□□□□□□□。

\section{適切な章立て}
□□□□□□□□□□□□□□□□□□□□□□□□□□□□□□□□□□□□□□□□□□□□□□□□□□□□□□
□□□□□□□□□□□□□□□□□□□□□□□□□□□□□□□□。
□□□□□□□□□□□□□□□□□□□□□□□□□□□□□□□□□□□□□□□□□□□□□□□□□□□□□□
□□□□□□□。

\section{適切な章立て}
□□□□□□□□□□□□□□□□□□□□□□□□□□□□□□□□□□□□□□□□□□□□□□□□□□□□□□
□□□□□□□□□□□□□□□□□□□□□□□□□□□□□□□□。
□□□□□□□□□□□□□□□□□□□□□□□□□□□□□□□□□□□□□□□□□□□□□□□□□□□□□□
□□□□□□□。

\begin{figure}
\centering
\includegraphics{image/hungry.pdf}
\caption{お腹をすかせた少年}
\label{fig:whole}
\end{figure}

\section{おわりに}
□□□□□□□□□□□□□□□□□□□□□□□□□□□□□□□□□□□□□□□□□□□□□□□□□□□□□□
□□□□□□□□□□□□□□□□□□□□□□□□□□□□□□□□。
□□□□□□□□□□□□□□□□□□□□□□□□□□□□□□□□□□□□□□□□□□□□□□□□□□□□□□
□□□\cite{Sou}。

\bibliography{mybib}
\end{document}
